\section{Resumen}

En este laboratorio consistió en desarrollar un sistema de control por voz, mediante el uso del microcontrolador Arduino Nano 33 BLE y el programa Edge impulse, el cual es un programa para entrenar una red neuronal con palabras calves, en el cual se utilizaron 3, una para simular que se controlan unas luces, se utilizó ``lumos'', para simular que se controla un reproductor de música, se utilizó ``música'' y para simular que se controla un televisor, se utilizó ``tele''. Por medio del micrófono que esta incorporado en el Arduino Nano 33 BLE se da la instrucción de cual de las 3 palabras se quiere utilizar. También se implementó el uso de IOT (Intenet de las Cosas), y el motivo es que se conectó el dispositivo a la red, con el fin de transferir datos y visualizarlos en forma de grafica en la plataforma Thingsboard. En la parte de machine learning, se agregaron además de las 3 palabras calves, se agregó ruido y unknown, esto con el fin de que el entrenamiento sea mejor y mas certero a la hora de ejecutar el programa.
