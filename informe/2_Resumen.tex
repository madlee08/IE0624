\section{Resumen}

En este reporte se realizó un sismógrafo digital con el fin de medir, registrar y estudiar las oscilaciones en el edificio de Ingeniería Eléctrica. El diseño de este sismógrafo se realizó por medio de un microcontrolador STM32F429 y con la biblioteca libopencm3. La alimentación de dicho sismógrafo se da por medio de paneles solares que cargan constantemente una batería de 9V. Por medio del giroscopio del STM32F429 se puede realizar lecturas en las 3 coordenadas, es decir, en el eje \textit{X}, \textit{Y} y \textit{Z}. Se implemento el uso de IOT (Intenet de las Cosas), y el motivo es que se conectó el dispositivo a la red, con el fin de transferir datos y visualizarlos en la plataforma Thingsboard. Se implemento es uso de un switch que habilitara o deshabilitara el uso de una comunicación con el puerto USART, y se utilizó uno de los diodos LED's que tiene el STM32F429 para indicar que se esta transmitiendo datos. También se realizo la visualización del giroscopio, el nivel de la batería y si el USART esta activado o no en la pantalla LCD del STM32F429. Se realizó un script en Python para leer/escribir al puerto serial/USB y enviar información del giroscopio y el nivel de batería al dashboard de la plataforma Thingsboard.

