\section{Resumen}

En este laboratorio el microcontrolador que se utilizó fue el Arduino UNO. Se realizó un diseño de manera que haya una conversión de tensión, ya que el sistema total recibe 4 canales de tensiones que se encuentran en el rango $[\SI{-24}{\volt}, \SI{24}{\volt}]$ y se convirtió en tensiones entre el rango de $[\SI{0}{\volt}, \SI{5}{\volt}]$ así los ADC del microcontrolador Arduino puede manejarlo sin dañarse. Además, se implemento un switch para calcular las tensiones tanto en AC como en DC. Además, para cada canal de tensión se implementó un LED rojo para alertar que el voltímetro midió una tensión con magnitud superior a \SI{20}{\volt}. También, se implementó una pantalla LCD para poder imprimir en esa pantalla los valores finales tomados por el Arduino y comunicación serial para enviar las lecturas por ese protocolo y ser recibidos por un script de Python para ser guardados en un archivo csv. La conclusión principal es que el laboratorio fue implementado exitosamente pues se cumplió con todos los requerimientos del enunciado.




