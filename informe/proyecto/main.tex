\documentclass[12pt,a4paper]{article}
\usepackage[utf8]{inputenc}%Para Tildes y ñ%
\usepackage[spanish]{babel}
\usepackage{amsmath}
\usepackage{amsfonts}
\usepackage{amssymb}
\usepackage{adjustbox}
\usepackage{graphicx} 
\usepackage{pdfpages} %para importar paginas de un pdf 
\usepackage{booktabs}
\usepackage{hyperref} 
\usepackage{geometry} 
\usepackage{multirow}
\usepackage{float}		% Para ubicar las tablas y figuras justo después del texto
\usepackage{pdfpages}
\usepackage{enumerate}%listas y viñetas
\usepackage{xcolor}
\usepackage{siunitx}
\usepackage{listings}
\usepackage{titling}
\usepackage{subcaption}
\usepackage{placeins}

\decimalpoint
\addto\captionsspanish{\renewcommand{\listtablename}{Índice de tablas}}	% Cambiar nombre a lista de tablas 
\addto\captionsspanish{\renewcommand{\tablename}{Tabla}} % Cambiar nombre a tablas
\setlength{\droptitle}{-1in} 
\geometry{a4paper, margin=1in}

\hypersetup{
    colorlinks=true,
    linkcolor=orange,
    urlcolor=blue,
    citecolor=cyan,
    pdftitle={IE0624 Laboratorio de microcontroladores | Proyecto | Mike Mai Chen - Javier Solera Bolaños},
    pdfpagemode=FullScreen,
}


\renewcommand{\tt}[1]{\texttt{#1}}
\renewcommand{\it}[1]{\textit{#1}}
\renewcommand{\bf}[1]{\textbf{#1}}

%---------------------------------------
\author{\small Estudiantes\\Javier Solera Bolaños | B66963\\Mike Mai Chen | B94487\\\vspace*{0.5in}\\\small Profesor\\Marco Villalta Fallas\vspace*{1.35in}}

\title{Universidad de Costa Rica\\{\small Facultad de Ingeniería\\Escuela de Ingeniería Eléctrica\\IE-0624 Laboratorio de Microcontroladores\\II ciclo 2023\\\vspace*{0.55in} Proyecto}\\ Diseño de un medidor de potencia \& factor de potencia con Arduino UNO para aplicaciones de baja potencia \vspace*{1.35in}}

\date{6 de Diciembre de 2023} 


%---------------------------------------
\begin{document} 
\maketitle  
\thispagestyle{empty}%%no numerar la portada
\renewcommand{\thepage}{\roman{page}}
\newpage
{
    \hypersetup{linkcolor=black}
    \tableofcontents
}
\newpage
%\listoffigures 
%\newpage
%\listoftables  
%\newpage
%%%%%%%%%%  
\renewcommand{\thepage}{\arabic{page}} 
\setcounter{page}{1}

\newpage  
\section{URL del Repositorio de GitHub}

El repositorio de git donde se trabajó todo lo referente al laboratorio se encuentra en \href{https://github.com/madlee08/IE0624/tree/lab-05}{Github}.

\section{Resumen}

En este laboratorio el microcontrolador que se utilizó fue el Arduino UNO. Se realizó un diseño de manera que haya una conversión de tensión, ya que el sistema total recibe 4 canales de tensiones que se encuentran en el rango $[\SI{-24}{\volt}, \SI{24}{\volt}]$ y se convirtió en tensiones entre el rango de $[\SI{0}{\volt}, \SI{5}{\volt}]$ así los ADC del microcontrolador Arduino puede manejarlo sin dañarse. Además, se implemento un switch para calcular las tensiones tanto en AC como en DC. Además, para cada canal de tensión se implementó un LED rojo para alertar que el voltímetro midió una tensión con magnitud superior a \SI{20}{\volt}. También, se implementó una pantalla LCD para poder imprimir en esa pantalla los valores finales tomados por el Arduino y comunicación serial para enviar las lecturas por ese protocolo y ser recibidos por un script de Python para ser guardados en un archivo csv. La conclusión principal es que el laboratorio fue implementado exitosamente pues se cumplió con todos los requerimientos del enunciado.





\section{Objetivos}
\subsection{Objetivo general}
Diseñar un medidor de potencia y factor de potencia con un Arduino UNO para aplicaciones de baja potencia.

\subsection{Objetivos específicos}
\begin{itemize}
    \item Diseñar un voltímetro que se ajuste a la entrada de tensión  del arduino.
    \item Conseguir un sensor de corriente que devuelva las lecturas como tensiones de 0 V a 5 V.
    \item Calcular la potencia consumida mediante software.
    \item Calcular la factor de potencia  mediante software.
    \item Diseñar el software para mostrar las mediciones en una pantalla.
    \item Utilizar una pantalla LCD para mostrar los valores de tensión, corriente, potencia y factor de potencia.
    \item Verificar el funcionamiento del proyecto mediante simulaciones.
    \item Diseñar los valores de los componentes a utilizar.
    \item Poner a prueba el proyecto en la vida real.
\end{itemize}
\section{Alcances}
Por un lado, el medidor de potencia medirá en DC con tensiones entre -20 a 20 V y corrientes entre -5 A a 5 A.
Por otro lado, en AC el medidor de potencia medirá tensiones no mayor a  14.14 V RMS y corrientes no mayor a  3.53 A RMS.

\section{Justificación}
En la carrera de Ingeniería Eléctrica los estudiantes eventualmente llevarán el curso de Laboratorio de electrónica I donde tienen acceso y conocimiento de equipos de medición como el multímetro y osciloscopio mientras diseñan sus circuitos y proyectos. Por lo tanto, no es de extrañar que algunos estudiantes quieran realizar proyectos personales en sus casas. Por temas de seguridad, es importante que constantemente monitoricen sus proyectos personales los equipos de medición. Sin embargo, el adquirir estos equipos de medición son costosos. Es aquí donde entra en juego este proyecto, que pretende ofrecer de una manera económica un equipo de medición para dicho proyectos personales.
\section{Nota teórica}
\subsection{Características del Arduino Nano 33 BLE}
\subsubsection{Características generales}
El Arduino Nano 33 BLE Sense es una placa de desarrollo de baja potencia que trae el nRF52840 de Nordic Semiconductor. Este procesador es un ARM Cortex-M4 @ \SI{64}{\mega\hertz} de 32 bits \cite{nano33, nrf}. Adicionalmente, la placa de desarrollo cuenta con
\begin{itemize}
    \item \SI{1}{\mega\byte} de memoria de almacenamiento,
    \item \SI{256}{\kilo\byte} de memoria RAM,
    \item UART,
    \item SPI,
    \item Comunicación I2C,
    \item Bluetooth Low Energy,
    \item módulo LSM9DS1 (acelerómetro, giroscopio y magnetómetro),
    \item módulo MP34DT05 (micrófono),
    \item módulo APDS9960 (sensor de proximidad, luz y gestos),
    \item módulo LPS22HB (barómetro) y
    \item módulo HTS221 (sensor de temperatura y humedad) \cite{nano33}.
\end{itemize}

\subsubsection{Características eléctricas}
Los rangos absolutos que se deben respetar para el microcontrolador de la placa de desarrollo son
\begin{itemize}
    \item tensión operación máxima $V_\text{DD max}$: \SI{3.3}{\volt},
    \item tensión máxima de entrada $V_\text{in max}$: \SI{21}{\volt},
    \item corriente máxima admitida para aplicaciones del usuario: \SI{950}{\milli\ampere} \cite{nano33}.
\end{itemize}
    

\subsubsection{Diagrama de bloques}
La figura \ref{mcu-diagram} ilustra el diagrama de bloques del Arduino Nano BLE 33 Sense. Los bloques de interés para este laboratorio son el propio CPU así como el micrófono (módulo MP34DT05) \cite{nano33}. El resto de bloques de la placa no serán utilizados para este laboratorio.

\begin{figure}[H]
    \centering
    \includegraphics[width=\textwidth]{Documentos/NANO33BLE_V2.0_sch.pdf}
    \caption{Diagrama de bloques del CPU nRF52840. Fuente y créditos: \cite{nano33}.}
    \label{mcu-diagram}
\end{figure}


\subsubsection{Diagrama de pines}

\begin{figure}[H]
    \centering
    \includegraphics[width=\textwidth]{Imagenes/Arduino_Nano.png}
    \caption{Diagrama de pines del Arduino Nano BLE 33 Sense. Fuente y créditos: \cite{nano33}.}
    \label{pinout}
\end{figure}
\section{Diseño del circuito} 
\subsection{Esquemático}

Para el diseños de hardware se utilizo el circuito que se muestra en la figura \ref{Fig: Diseño hardware} y en la figura \ref{Fig: Diagrama_hardware} se puede observar el diagrama eléctrico que se siguió. De manera que para la conversión de tensiones de [0V, 9V] a [0V, 3V], por otro lado se utilizo un switch con el fin de controlar la comunicación en \textit{USART}.

\begin{figure}[H]
\centering
\includegraphics[width=0.9\textwidth]{Imagenes/Circuito.jpg} 
\caption{Diseño del circuito.}
\label{Fig: Diseño hardware}
\end{figure}

\begin{figure}[H]
\centering
\includegraphics[width=0.9\textwidth]{Imagenes/Diagrama_electricoLab4.jpg} 
\caption{Diagrama Eléctrico del Circuito}
\label{Fig: Diagrama_hardware}
\end{figure}


\subsection{Disminución del efecto rebote de los switches}
De acuerdo con la información de referencia del STM32F429, el microcontrolador tiene resistencias de \it{pull-up} internas de \SI{40}{\kilo\ohm} \cite{stm32micro}.
La obtención de estos valores de resistencias y el valor de la capacitancia se da por medio de la ecuación
\begin{equation}
    \tau = RC.
\end{equation}
Como el interruptor oscila entre estar cerrado y abierto entre diez a cien veces en un periodo de \SI{1}{\milli\second}, se puede llegar a despejar la capacitancia $C$
\cite{boton},
\begin{equation*}
    C = \frac{\tau}{R} = \frac{\SI{1}{\milli\second}}{\SI{40}{\kilo\ohm}} = \SI{25}{\nano\farad}.
\end{equation*}

Por lo tanto, se decide utilizar una capacitancia de \SI{22}{\nano\farad}, ya que es el valor comercial más cercano.
 

 
\subsection{Cálculo de las resistencias}
En la figura \ref{Fig: cal_resist}, se ilustra la red resistiva que ``convierte'' tensiones entre \SI{0}{\volt} a \SI{9}{\volt} a un rango entre \SI{0}{\volt} a \SI{3}{\volt}. Se destaca que a la hora implementarlo, se decidió utilizar un potenciómetro en lugar de la resistencia \textit{$R_2$}, el motivo es porque así se puede manipular la tensión de salida y probar la condición de cuando la tensión llega a valores cercanos a \SI{7}{\volt}. El calculo de las resistencia se observa a continuación
\begin{align}
    V_o &= \frac{R_2 \cdot V_\text{in}}{R_1 + R2}\\
\end{align}
siendo $R_1$ un valor arbitrario. Por lo tanto, para un $R_1 = \SI{1}{\kilo\ohm}$, $R_2$ debe de ser
\begin{align*}
    R_2 &\approx \SI{1250}{\kilo\ohm}.
\end{align*}
Se escogerá un potenciómetro de \SI{5}{\kilo\ohm} que va a sustituir la resistencia $R_2$.

\begin{figure}[H]
\centering
\includegraphics[width=0.6\textwidth]{Imagenes/divisordetension.png} 
\caption{Red resistiva para convertir el rango de la tensión de entrada a un rango apto para el STM32F429. Fuente y créditos: \cite{divisor}.}
\label{Fig: cal_resist}
\end{figure}


\section{Diseño de los programas}
\subsection{Diagrama de flujo del osciloscopio}
El osciloscopio se puede implementar en el Arduino UNO siguiendo el diagrama de flujo que se ilustra en la figura \ref{fsm}. Por un lado, en AC se debe muestrear múltiples veces para determinar el valor máximo de una señal senoidal. Luego el proceso anterior se debe repetir varias veces tal que se obtiene el último valor máximo así como valores máximos pasados. Esto permite que ante un cambio en la amplitud de la señal senoidal, en especial una disminución, el medidor puede registrar el cambio y mostrar el valor correcto en pantalla.
Por otro lado, en DC también se conserva varias lecturas anteriores además de la lectura más reciente, esto para calcular el valor promedio.
\begin{figure}[H]
    \centering
    \includegraphics[width=14cm]{Imagenes/fsm.pdf}
    \caption{Diagrama de flujo del programa en el Arduino UNO.}
    \label{fsm}
\end{figure}

Ahora, en la figura \ref{fsm-py} se ilustra el diagrama de flujo del programa utilizado para recibir las lecturas por comunicación serial y guardarlo en un archivo csv. La implementación para configurar y abrir el puerto serial en Python se puede consultar en \cite{serial}.
\subsection{Diagrama de flujo del capturador de datos}
\begin{figure}[H]
    \centering
    \includegraphics[width=10cm]{Imagenes/fsm_py.pdf}
    \caption{Diagrama de flujo del programa que captura y guarda las lecturas de tensiones.}
    \label{fsm-py}
\end{figure}
\section{Lista de equipos}




%%
%La lista de componentes utilizados en el experimento se muestra en la Tabla 2.
%\begin{table*} [!ht]
%\caption{Lista de componentes}
%\label{t3}
%\begin{center}
%\begin{tabular}{r|ccccc}
%\hline
%\textbf{Componente} & \textbf{Sigla} & \textbf{Valor nominal} & \textbf{Valor real}& \textbf{Potencia}\\ 
%\hline
%Resistor  & $R_1$ & 1 k$\Omega$ &946.7 $\Omega$ & 0.25 W\\  
%Resistor  & $R_2$ & 2 k$\Omega$ & 2.04 k$\Omega$  & 0.25 W\\  
%Potenciómetro  & P1 & 10 k$\Omega$  & 5.27 k$\Omega$ & 0.25 W\\  
%Potenciómetro  & $P2$ & 5 k$\Omega$  & 111.13  $\Omega$ & 0.25 W\\  
%Capacitor &$C_{aux}$& 100 nF &97.8 nF& $-$\\ 
%Capacitor &Cx& 100 nF &92.8 nF& $-$\\ 
%Diodo LED (5) &$-$ &$-$ &$-$ & $-$\\ 
%Acelerómetro ADXL335& ADXL335&$-$ &$-$ & $-$\\ 
%LM555 (2)  & LM555 &$-$ &$-$ & $-$\\  
%CD4017   & CD4017 &$-$ &$-$ & $-$\\   
%LM311   & LM311 &$-$ &$-$ & $-$\\  


%\hline
%\end{tabular}
%\end{center}
%\end{table*}





\begin{table}[H]
\centering
\resizebox{\textwidth}{!}{%
\begin{tabular}{|c|c|c|c|}
\hline
\textbf{Componentes} & \textbf{Valor Nominal} & \textbf{Cantidades} & \textbf{Precio en el Mercado} \\ \hline
\multirow{3}{*}{Resistencias} & 1.8 k & 8 & \$ 0.12 \\ \cline{2-4} 
 & 270 & 8 & \$ 0.12 \\ \cline{2-4} 
 & 820 & 4 & \$ 0.12 \\ \cline{2-4} 
 & 10 K & 4 & \$ 0.12 \\ \hline
Capacitor & 0.1 F & 1 & \$ 0.12 \\ \hline
Diodos LED's Rojo & - & 4 & \$ 0.55 \\ \hline
Botón & - & 2 & \$ 0.20 \\ \hline
Microcontrolador Arduino UNO & - & 1 & \$ 49.95 \\ \hline
\end{tabular}%
}
\caption{Lista de Componentes }
\label{tab:Lista_Componentes}
\end{table}
\section{Resultados y análisis}


\subsection{Verificación de los valores del giroscopio, batería y USART en la pantalla LCD}

Se realizaron pruebas por separado para poder verificar el correcto funcionamiento del giroscopio, de la batería y del USART.

\subsubsection{Verificación de la lectura del giroscopio}

    \begin{figure}[H]
        \begin{subfigure}{0.5\textwidth}
        \centering
        \includegraphics[width=\textwidth]{Imagenes/giroscopio2.png} 
        \caption{Prueba del giroscopio sobre la mesa sin movimiento.}
        \label{Fig:giroscopio}
    \end{subfigure}
    \begin{subfigure}{0.5\textwidth}
        \centering
        \includegraphics[width=\textwidth]{Imagenes/giroscopio.jpg} 
        \caption{Prueba del giroscopio levantándolo y con movimiento.}
        \label{Fig:giroscopio2}
    \end{subfigure}
    \end{figure}

En la figura \ref{Fig:giroscopio}, se puede observar que esta prueba fue cuando el microcontrolador estaba sobre la mesa, al estar en la mesa los valores que se obtienen son cercanos a 0, no son cero ya que es probable que el giroscopio sea muy sensible ante cualquier movimiento, aunque sea muy pequeño, por esta razón siempre marca algún valor aunque parezca que no esta en movimiento, y en la figura \ref{Fig:giroscopio2} se puede observar cómo la lectura del giroscopio cambia al someterse a un movimiento, en este caso se levantó de la mesa, demostrando así que los valores de \textit{X}, \textit{Y} y \textit{Z} cambian y así poder simular un sismógrafo.
Adicionalmente, en la entrega de este reporte se incluye tres videos, donde cualquiera de los tres videos se observa que se realiza la lectura del giroscopio correctamente.

\subsubsection{Verificación de la lectura de la Batería}

    \begin{figure}[H]
        \begin{subfigure}{0.5\textwidth}
        \centering
        \includegraphics[width=\textwidth]{Imagenes/Prueba_Bat.jpg} 
        \caption{Prueba Batería en rango lejos de 7V.}
        \label{Fig:Prueba_Bat}
    \end{subfigure}
    \begin{subfigure}{0.5\textwidth}
        \centering
        \includegraphics[width=\textwidth]{Imagenes/Prueba_Bat2.jpg} 
        \caption{Prueba Batería en rango cercano a 7V.}
        \label{Fig:Prueba_Bat2}
    \end{subfigure}
    \end{figure}

Como se observan en las figuras \ref{Fig:Prueba_Bat} y \ref{Fig:Prueba_Bat2}, se realizó una configuración de un divisor de tensiones con un potenciómetro con el fin de poder simular una caída de tensión de la batería, y conectando el pin PA0 a la salida de ese circuito y protegiendo así el STM32F429, entonces cuando el microcontrolador lee de la batería 8V o más, el cual se mantiene lejos del limite de operación del microcontrolador, el LED rojo que tiene incorporado el microcontrolador se encuentra apagado. Luego, se ajusta el potenciómetro de manera que lee una tensión por debajo de 8V y cerca a la tensión de operación, entonces el LED rojo se enciende y parpadea, indicando que se está trabajando con una tensión muy cercana a la tensión de operación del STM32F429 como se observa en la figura \ref{Fig:Prueba_Bat2}.


\subsubsection{Verificación de la lectura de la USART}

\begin{figure}[H]
    \begin{subfigure}{0.5\textwidth}
    \centering
    \includegraphics[width=\textwidth]{Imagenes/Prueba_USART.jpg} 
    \caption{Prueba USART en estado ``OFF''}
    \label{Fig:Prueba_USART}
\end{subfigure}
\begin{subfigure}{0.5\textwidth}
    \centering
    \includegraphics[width=\textwidth]{Imagenes/Prueba_USART2.jpg} 
    \caption{Prueba USART en estado ``ON''}
    \label{Fig:Prueba_USART2}
\end{subfigure}
\end{figure}

En las figuras \ref{Fig:Prueba_USART} y \ref{Fig:Prueba_USART2}, se observan las pruebas que se realizaron para verificar que cuando el switch se encuentra en ``OFF'', el microcontrolador lee que ese PIN esta en bajo, es decir, la comunicación USART esta desactiva como se observa en la pantalla LDC de la figura \ref{Fig:Prueba_USART}, cuando el switch esta en ``ON'', quiere decir que la el microcontrolador lee que el PIN se encuentra en alto y el LED verde del microcontrolador se enciende, es decir, que la comunicación USART se activa, como se observa en la pantalla LDC de la figura \ref{Fig:Prueba_USART2}. Para evitar el efecto rebote del switch, se le conectó un capacitor de 22 nF y no fue necesario conectarle una resistencia ya que este microcontrolador trae una resistencia interna según la hoja de datos \cite{stm32micro, datasheet}. 

\subsubsection{Verificación del funcionamiento de la Batería y el USART al mismo tiempo}

\begin{figure}[H]
    \begin{subfigure}{0.5\textwidth}
    \centering
    \includegraphics[width=0.9\textwidth]{Imagenes/Prueba_Bat_USART.jpg} 
    \caption{Prueba del funcionamiento de la USART y la Batería cercano a 7V.}
    \label{Fig:Prueba_Bat_USART}
\end{subfigure}
\begin{subfigure}{0.5\textwidth}
    \centering
    \includegraphics[width=\textwidth]{Imagenes/Prueba_Bat_USART2.jpg} 
    \caption{Prueba del funcionamiento de la USART y la Batería cercano a 7V.}
    \label{Fig:Prueba_Bat_USART2}
\end{subfigure}
\end{figure}

En las figuras \ref{Fig:Prueba_Bat_USART} y \ref{Fig:Prueba_Bat_USART2}, se observan como las 2 pruebas anteriores funcionan al mismo tiempo, cabe a destacar que las mangones que se muestran son únicamente cuando la tensión está cerca de la tensión de operación, es decir, esta cerca de los 7 V, de manera que el LED rojo se enciende y además se activa la comunicación USART, entonces los LED's parpadean de rojo a verde, comunicando que la comunicación USART esta activa y que la tensión de la batería esta cerca de la tensión de operación.



\subsection{Verificación que los datos se envíen correctamente por USART}
Ahora, se debe verificar que los datos se envíen correctamente por USART para ser mostrados en una terminal de la computadora. La figura \ref{usart-data-received} ilustra que sí se reciben los datos desde la placa de desarrollo y son mostrados correctamente en la terminal. En el video que se incluye en la entrega de este reporte, los videos \tt{demo\_usart\_off} ilustra que por más movimiento que se haga a la placa, éste no enviará los datos por USART ya que el USART está deshabilitado. En cambio, los videos \tt{demo\_usart\_iot\_on\_*} se muestra que cuando el USART está habilitado, los datos son enviados por comunicación serial y se reciben correctamente en la terminal.
\begin{figure}[H]
    \centering
    \includegraphics[width=12cm]{Imagenes/usart_data_received.png}
    \caption{Captura de pantalla de la terminal donde se reciben los datos por USART.}
    \label{usart-data-received}
\end{figure}

\subsection{Verificación de los datos se reflejen en iot.eie.ucr.ac.cr}
Luego, se debe comprobar que los datos recibidos por USART se envíen correctamente a \tt{iot.eie.ucr.ac.cr}. La figura \ref{iot-data-received} ilustra que los datos son enviados correctamente a Thingsboard y se muestran correctamente en el dashboard. Los videos \tt{demo\_usart\_iot\_on\_*} incluidos en la entrega ilustran que efectivamente los datos llegan correctamente a Thingsboard puesto que la página web se actualiza cuando se envía datos desde la placa de desarrollo.
\begin{figure}[H]
    \centering
    \includegraphics[width=\textwidth]{Imagenes/iot_data_received.png}
    \caption{Captura de pantalla de la terminal donde se envían a \tt{iot.eie.ucr.ac.cr} los datos.}
    \label{iot-data-received}
\end{figure}


\subsection{Verificación de las tensiones se encuentren dentro del rango seguro}

Con ayuda de un voltímetro se verificó que las tensiones que lee el microcontrolador se encuentran en rango seguro según la hoja del fabricante, y así evitar que se queme algún componente del microcontrolador \cite{stm32micro, datasheet}.


\begin{figure}[H]
    \begin{subfigure}{0.5\textwidth}
    \centering
    \includegraphics[width=\textwidth]{Imagenes/BATERIA.jpg} 
    \caption{Lectura de la tensión que lee del PIN PA0 (la batería) el microcontrolador}
    \label{Fig:BATERIA}
\end{subfigure}
\begin{subfigure}{0.5\textwidth}
    \centering
    \includegraphics[width=\textwidth]{Imagenes/USART.jpg} 
    \caption{Lectura de la tensión que lee del PIN PA2 el microcontrolador}
    \label{Fig:USART}
\end{subfigure}
\end{figure}

Como se puede ver, en las figuras \ref{Fig:BATERIA} y \ref{Fig:USART} se observa que el voltaje que marca son 2.92V para la etapa de la batería y para la etapa del switch que controla el USART marca 2.47V, según la hoja del fabricante el valor máximo de tensión que soporta los pines es 4 V según la hoja del fabricante \cite{stm32micro, datasheet}, y ambos valores están por debajo de esa de esa tensión, por lo tanto se concluye que se está trabajando con unas tensiones seguras para el microcontrolador y así no dañar el STM32F429.





































\section{Conclusiones y recomendaciones}
\begin{itemize}
    \item El laboratorio se concluyó de manera satisfactoria ya que se cumplió con todas las especificaciones del enunciado.
    \item Esto incluye crear un modelo de red neuronal capaz de clasificar los comandos deseados, así como cargar el modelo en el Arduino.
    \item Además, se pudo implementar la redirección de salida de los comandos con un valor numérico asociado a un script de Python y reenviarlos a \tt{iot.eie.ucr.ac.cr}, así como guardarlo en un archivo de texto.
    \item A pesar de que se debe repetir varias veces los comandos, el Arduino eventualmente clasifica correctamente el comando y lo envía a Thingsboard.
\end{itemize}



 






\newpage


\bibliographystyle{IEEEtranS}
\bibliography{bibliografia.bib}
\newpage

\newpage

\section{Anexos}

\includepdf[pages={1-3, 8, 10-11, 14-15}]{Documentos/arduino.pdf}
% \includepdf[pages={1}]{Documentos/capacitor.pdf}

  
  

\end{document}
 